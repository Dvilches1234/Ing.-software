\documentclass{udpreport}
%\headertext{Metodos Numéricos}
\title{Commercial off-the-shelf}
\author{Manuel Tobar, Rocio Venegas, Diego Vilches.}
\usepackage{amssymb}
\usepackage{amsmath}
\usepackage{graphicx}
\usepackage{float}
\usepackage{array}
\udpschool{Escuela de Informática y telecomunicaciones}
\usepackage{listings}
\usepackage{color}
\usepackage{hyperref}

\definecolor{dkgreen}{rgb}{0,0.6,0}
\definecolor{gray}{rgb}{0.5,0.5,0.5}
\definecolor{mauve}{rgb}{0.58,0,0.82}
\begin{document}
\maketitle
\tableofcontents
\chapter{Introducción}

\chapter{Metodología}
	\section{Explicación}
	COTS o Componente sacado del estante (Commercial Off-The-Shelf) hace referencia a los tipos de aplicaciones de software que están diseñados con la intención de una funcionalidad en específico. Estas aplicaiones, tienen la características que pueden ser compradas en el mercado, estando así, a la disposición del público general. Debido a esto, la mayoría incluye características y funciones que lo hacen reutilizable. Ejemplo de esto, son los gestores de bases de datos y los antivirus.
	\\
	
	
	Actualmente, no es raro que los sistemas grandes permitan acceder a sus funciones mediante Interfaces de Programación de Aplicacipones (APIs). Por este motivo, la creacion de grandes sistemas, utilizando sistemas COTS más pequeños debería siempre ser una opción, ya que al utilizarlos, se podrían reducir costes y tiempos en comparación al desarrollo completo de un nuevo software.
	\\
	
	
	Si bien los sistemas COTS son lo suficientemente flexibles como para permitir modificaciones, hacer demaciados cambios puede ser riesgoso, ya que después de un cierto punto el vendedor del software ya no podría seguir ofreciendo soporte o actualizaciones. Esto significa, que ahora la responsabilidad de mantener el software actualizado pasa a ser de quien compró y modificó el producto en vez de del vendedor.
	%Rellenar acá
	\\	
	
	
	Como todas las metodologías, esta tiene sus ventajas y desventajas, las cuales dependar del contexto en el que se esté trabajando. Estas, serán exploradas en las siguientes secciones.
	\section{Ventajas}
	\begin{itemize}
		\item \textbf{Tiempo de desarrollo:} como todos los componentes ya están programados y probados, el tiempo de desarrollo se ve disminuído. De esta manera, hay más tiempo para entender los requisitos, configurar el sistema completo y probarlo.
		
		\item \textbf{Fallas:} es esperable que un producto COTS tenga menos defectos, debido a que este ha sido sometido, no solo a rigurosas pruebas, si no también, a la retroalimentación de otros usuarios del sistema. 
		
		\item \textbf{Costos:} como los costos asociados al producto están siendo pagados por todos los clientes, el precio para un cliente singular se ve reducido en comparación a los costos asociados a desarrollar y mantener un software.
		
		\item \textbf{Soporte y entrenamiento:} el vendedor del software proveera entrenamiento para que los usuarios aprendan a utilizar el sistema. Es díficil lograr este nivel de servicio para software hecho a la medida. De forma similiar, el soporte  puede ser entregado con mayor eficiencia.
		
	\end{itemize}
	\section{Desventajas}
	\begin{itemize}
	\item \textbf{Funcionalidad:} al ocupar un producto COTS, es posible que las funcionalidades de este fallen o tengan un rendimiento pobre dada una situación específica. También es posible, que el sistema tenga funcionalidades que son irrelevantes para el proyecto. La existencia de estas, podría obligar a trabajar en una forma no óptima. Esta clase de problemas, no es responsabilidad del vendedor del software, por lo que recae en manos del desarrollador solucionarlos.
	
	\item \textbf{Interoperabiliad del sistema COTS:} es posible que sea dificil que varios productos COTS puedan operar en conjunto, ya que, inicialmente cada producto tiene una forma esperada de uso. Arreglar este tipo de problemas, podría resultar en aumentar considerablemente la cantidad de trabajo.
	
	\item \textbf{Control sobre la evolución del sistema:} los cambios hechos en el productos son decidios por los vendedores. Estos cambios pueden producir problemas como incompatibilidad con versiones anteriores, funcionalidades nuevas no deseadas y en algunos casos, la carencia de soporte de versiones previas. 
	
	\item \textbf{Soporte de los vendedores:} si bien, los vendedores pueden ofrecer soporte cuando surgen problemas, los cambios en el mercado pueden ser una dificultad. Por ejemplo, si un producto tiene una escasa demanda, el vendedor puede decidir dejar de venderlo, lo que conllevaría al final de su soporte. 
	\end{itemize}

\chapter{Aplicación a cachupín}

\chapter{Conclusión}
\chapter{Bibliografía}
\url{http://jezlister.com/cots/cots-commercial-off-the-shelf-lifecycle-model/}


\url{https://www.resqsoft.com/basics-cots-%E2%80%93-commercial-off-the-shelf-software.html}

Ingeniería del Software 7ma. Ed - Ian sommerville- Sección 18.5.1
\end{document}
